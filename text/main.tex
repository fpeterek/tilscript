\documentclass{article}
\usepackage[utf8]{inputenc}
\usepackage{xurl}
\usepackage[czech]{babel}
\usepackage{csquotes}

\MakeOuterQuote{"}

\setlength{\parskip}{\baselineskip}
\setlength{\parindent}{0pt}

\title{Kontrola typové koherence v jazyce TIL-Script}
\author{Filip Peterek}
\date{1. prosinec 2021}

\begin{document}

\maketitle

\section{Transparentní intenzionální logika}

Transparentní intenzionální logika~\cite{til-duzi} (dále také pouze TIL) je logický systém sloužící
k analýze přirozeného jazyka. TIL vychází z typovaného lambda kalkulu, a využívá rigorózně definovaný
typový systém za účelem zabránění velké škále chyb. Mezi chyby, které lze typovým systémem zachytit
patří například primitivnější chyby, jako je špatné pořadí argumentů (rovnice počítá Karla), ale také
chybná změna supozice způsobená inkorektní analýzou (funkce očekává úřad, avšak na vstupu dostane
individuum). Kontrolu typové koherence však lze provádět strojově. Tato práce se zabývá automatizací
typové kontroly, jež je užitečná převážně v případě, kdy analyzujeme složité TIL konstrukce a manuální
typová kontrola by tedy byla náchylná k chybě.

\section{TIL-Script}

TIL-Script~\cite{til-script} je funkcionální programovací jazyk sloužící k práci s TIL konstrukcemi.
Syntax TIL-Scriptu vychází z notace TIL, kterou však upravuje pro potřeby počítačového zápisu -- zatímco
TIL ve své notaci hojně využívá řecké abecedy nebo horních i dolních indexů, TIL-Script byl vytvářen
tak, aby byl jednoduše zapisovatelný i na běžné klávesnici. Z toho důvodu byl také TIL-Script zvolen
jako notace využívaná v této práci.

\section{Parser jazyka TIL-Script}

Parser byl vygenerován za využití generátoru parserů Antlr~\cite{antlr-src}. Díky volby již existujícího
jazyka jako notace programu stačilo gramatiku jazyka TIL-Script upravit do formátu zpracovatelného
Antlrem. Jelikož program pro kontrolu typové koherence využívá build systém Gradle, je parser automaticky
vygenerován během kompilace, je-li to potřeba. Výstup vygenerovaného parseru je poté převeden na vlastní
reprezentaci, která je méně generická a umožňuje ergonomičtější programatickou práci s abstraktním
syntaktickým stromem TIL-Script programu.

Ke kontrole syntaktických chyb je využita kontrola zabudovaná do nástroje Antlr. Nad rámec automatického
reportování zabudovaného v Antlru je implementován pouze vylepšený indikátor pozice chyby přímo
ve strojovém kódu, a kromě číselné pozice chyby je vypsán také samotný chybný řádek s graficky vyznačenou
chybou.

\begin{thebibliography}{50}

\bibitem{til-duzi}
Duží, M., Materna, P.: \textit{TIL jako procedurální logika} [cit. 2022-06-09] \\
Dostupné z: \url{http://www.cs.vsb.cz/duzi/aleph.pdf} \\
Marie Duží, Pavel Materna, 2012

\bibitem{til-script}
Ciprich, N., Duží, M., Košinár, M.: \textit{TIL-Script: Functional Programming Based on
Transparent Intensional Logic} [cit. 2022-06-09] \\
In: \textit{RASLAN 2007}, Sojka, P., Horák, A., (Eds.), Masaryk University Brno, 2007, pp. 37–42.

\bibitem{antlr-src}
Parr, T.: \textit{ANTLR v4} [cit. 2022-06-09] \\
Dostupné z: \url{https://www.antlr.org} \\
Terence Parr

\end{thebibliography}

\end{document}

